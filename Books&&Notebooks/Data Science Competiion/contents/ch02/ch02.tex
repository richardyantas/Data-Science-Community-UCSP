\documentclass[12pt]{book}
\usepackage{docmute}
\usepackage{import}
\usepackage{tikz}

\begin{document}

\chapter{Exploratory Data Analysis}
\section{Exploratory Data Analysis}

    Nowadays, most ecological research is donde with hypothesis testing and modelling in mind. 
    However , Exploratory Data Analysis(EDA), which uses visualization tools and computer synthetic descriptors, 
    is still required at the beginning of the statistical analysis of multidimensional data, in order to:

    \begin{itemize}
        \item Get an overview of the data
        \item Transform or recode some variables
        \item Orient further analyses
    \end{itemize}

    \subsection{Building Intuition about the data}
        \begin{enumerate}
            \item Getting domain knowledge
                It helps to deeper understand the problem
            \item CHecking if the data is intuitive
                To be agree with domain knowledge 
            \item Understanding how the data was generated
                A it is crucial to set up a poper Validation
        \end{enumerate}

    \subsection{Data cleaning}
        \begin{enumerate}
            \item Constant features
            \item Duplicated features
            \item Duplicated rows
            \item Check if datasets is shuffled
        \end{enumerate}

        here graphics!! :D 

    \subsection{Data splitting strategy}
        \begin{enumerate}
            \item                 
        \end{enumerate}
    \subsection{Exploratory data Analysis}
    \subsection{Exploratory anonymized data}
    \subsection{Validation and overfitting}
    \subsection{Validation strategies}
    \subsection{Visualizations}






        


%\subimport{img/}{img01}



\end{document}